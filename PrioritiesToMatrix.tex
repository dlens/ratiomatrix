\documentclass[10pt,twoside]{article}
%\usepackage{fontspec}
\usepackage{mathrsfs}
\usepackage{amsmath}
\usepackage{amsthm}
\newtheorem{lemma}{Lemma}
\newtheorem{note}{Note}
\makeindex
%\usepackage{xltxtra}

\usepackage{dljmv2overleaf}
\usepackage{dsfont}
\title{Priorities to Pairwise}
\author{Bill Adams}
\date{Aug 2017}
%\setcounter{tocdepth}{2}

\usepackage[hidelinks]{hyperref}


\begin{document}
\maketitle
%\tableofcontents

\begin{abstract}
Given a prioritiy vector $\vec{p}$ there is a method to create a pairwise comparison matrix $M_{\vec{p}}$ that has $\vec{p}$ as its priority vector.  In this paper we provide context for why this calculation could be useful, the definition of the calculation itself, and a few example calculations.
\end{abstract}

\section{Introduction}
Given $n$ items whose $n \times n$ pairwise comparison matrix is $M$, standard
AHP theory uses the largest eigenvector calculation to find the priority vector
$\vec{p}_M$ for those $n$ items.  In addition, standard AHP theory provides a way
to work in reverse; i.e. given a priority vector $\vec{p}$ we can create a
pairwise comparison matrix $M_{\vec{p}}$ whose largest eigenvector is $\vec{p}$.

This method of working in reverse can be useful in the following circumstance.
If one knows the priority vector for a group of users, but the software that gives
that priority does not provide the group pairwise comparison matrix, one can
reconstruct a pairwise comparison matrix for the group that results in the
given priority vector.  This circumstance is what prompted the writing of this
paper.

\section{Calculation and proof}
Given a priority vector $\vec{p}=(p_1, \ldots, p_n)$, we construct the matrix $M$
whose priority vector is $\vec{p}$ by taking the matrix
made of the ratios of the $p_i$'s.  We state this result as the following lemma.

\begin{note}
The reason we use the ratio matrix can be easily seen with an example.  If the 
priority vector is alt1=0.1, alt2=0.3, and alt3=0.6 we can easily see that
alt3 is 6 times better than alt1, because alt3 has a score of 0.6 and alt1 has
a score of 0.1 and the value of 6 is simply $0.6/0.1$ the ratio of alt3's score
to that of alt1's score.
\end{note}

\begin{lemma}
Let $\vec{p}=(p_1, \ldots, p_n)$ be the priority vector of $n$ items.  If we define
the $n\times n$ matrix $M_{\vec{p}}=(m_{ij})$ by the formula
$$m_{ij} = \frac{p_i}{p_j}$$
then the priority vector for $M_{\vec{p}}$ is $\vec{p}$.
\end{lemma}
\begin{proof}
Let largest eigenvector of $M_{\vec{p}}$ can be calculated from the following formula
$$\lim_{n \to \infty} \frac{b^n}{\sum_{i=1}^n b^n_i}$$
where $b^i$ is an $n$ dimensional vector defined using the following recursive formula:
\begin{eqnarray*}
	b^0 &=& (1, \ldots, 1) \\
	b^{i+1} &=& M_{\vec{p}} \cdot b^i
\end{eqnarray*}	
Thus we need only understand $b^i$.  Let us begin by calculating 
$b^1$.
\begin{eqnarray*}
	b^1 &=& M_{\vec{p}} \cdot b^0 \\
	&=&
	\begin{bmatrix}
	1 & p_1/p_2 & \ldots & p_1/p_n\\
	p_2/p_1 & 1 & \ldots & p_2/p_n \\
	\vdots & vdots & \ddots & \vdots \\
	p_n/p_1 & p_n/p_2 & \ldots & 1	
	\end{bmatrix}
	\begin{bmatrix}
	1 \\
	1 \\
	\vdots \\
	1
	\end{bmatrix} \\
	&=& \begin{bmatrix}
	\sum_{i=1}^n p_1/p_i	 \\
	\sum_{i=1}^n p_2/p_i	 \\
	\ldots \\
	\sum_{i=1}^n p_n/p_i	
	\end{bmatrix} \\
	&=& 
	\begin{bmatrix}
		p_1 \sum_{i=1}^n 1/p_i \\
		p_2 \sum_{i=1}^n 1/p_i \\
		\vdots \\
		p_n \sum_{i=1}^n 1/p_i
	\end{bmatrix} \\
	&=&\sum_{i=1}^n 1/p_i
	\begin{bmatrix}
	p_1 \\
	p_2 \\
	\ldots \\
	p_n	
	\end{bmatrix}
\end{eqnarray*}
Since we are dividing by the sum of the elements of $b^i$ in the limit, we can
multiply or divide $b^i$ by any constant and not change the limit.  Thus effectively
$b^1 = (p_1, \ldots, p_n)$.

Next we claim that $b^2 = (p_1, \ldots p_n)$.  If that is the case, then we know
that all of the $b^i$'s for $i\geq 1$ are that as well, and we will have shown that
the largest eigenvector of $M_{\vec{p}}$ is $\vec{p}$.  Let us calculate the value
of $b^2$ to complete the proof.

\begin{eqnarray*}
b^2 &=& M_{\vec{p}} \cdot b^1 \\
&=& 	
	\begin{bmatrix}
	1 & p_1/p_2 & \ldots & p_1/p_n\\
	p_2/p_1 & 1 & \ldots & p_2/p_n \\
	\vdots & vdots & \ddots & \vdots \\
	p_n/p_1 & p_n/p_2 & \ldots & 1	
	\end{bmatrix}
	\begin{bmatrix}
	p_1 \\
	p_2 \\
	\vdots \\
	p_n
	\end{bmatrix} \\
	&=&
	\begin{bmatrix}
	n p_1 \\
	n p_2 \\
	\vdots \\
	n p_n	
	\end{bmatrix}
\end{eqnarray*}
Again, since we are dividing by the sum of the elements of $b^2$ in the limit, we
can multiply or divide by any constant and not effect the limit calculation.  Thus, effectively, $b^2$ is effectively $(p_1, \ldots, p_n)$.
\end{proof}

\section{Examples}
We calculate the matrix $M_{\vec{p}}$ for several priority vectors $\vec{p}$ below.
\begin{enumerate}
	\item If $\vec{p} = (1/2, 1/3, 1/6)$, the ratio matrix $M_{\vec{p}}$ is:
	\begin{eqnarray*}
	M_{\vec{p}} &=&
	\begin{bmatrix}
		1 & \frac{1/2}{1/3} & \frac{1/2}{1/6} \\
		\frac{1/3}{1/2} & 1 & \frac{1/3}{1/6} \\
		\frac{1/6}{1/2} & \frac{1/6}{1/3} & 1
	\end{bmatrix} \\
	&=&
	\begin{bmatrix}
		1 & 3/2 & 3 \\
		2/3 & 1 & 2 \\
		1/3 & 1/2 & 1
	\end{bmatrix}
	\end{eqnarray*}
	\item If $\vec{p} = (3, 2, 1, 4)$ we can calculate the ratio matrix as below.  Notice that the priority vector need not be normalized (i.e. its components need not sum to 1).
	\begin{eqnarray*}
		M_{\vec{p}} &=&
		\begin{bmatrix}
			1 & \frac{3}{2} & \frac{3}{1}& \frac{3}{4} \\
			\frac{2}{3}& 1 & \frac{2}{1}& \frac{2}{4} \\
			\frac{1}{3} & \frac{1}{2} & 1 & \frac{1}{4} \\
			\frac{4}{3} & \frac{4}{2} & \frac{4}{1} & 1
		\end{bmatrix} \\
		&=&
		\begin{bmatrix}
			1 & \frac{3}{2} & 3 & \frac{3}{4} \\
			\frac{2}{3} & 1 & 2 & \frac{1}{2} \\
			\frac{1}{3} & \frac{1}{2} & 1 & \frac{1}{4} \\
			\frac{4}{3} & 2 & 4 & 1
		\end{bmatrix}
	\end{eqnarray*}
\end{enumerate}

\end{document}
